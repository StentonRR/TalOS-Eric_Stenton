%%%%%%%%%%%%%%%%%%%%%%%%%%%%%%%%%%%%%%%%%
%
% CMPT 424N-111
% Fall 2019
% Lab One
%
%%%%%%%%%%%%%%%%%%%%%%%%%%%%%%%%%%%%%%%%%

%%%%%%%%%%%%%%%%%%%%%%%%%%%%%%%%%%%%%%%%%
% Short Sectioned Assignment
% LaTeX Template
% Version 1.0 (5/5/12)
%
% This template has been downloaded from: http://www.LaTeXTemplates.com
% Original author: % Frits Wenneker (http://www.howtotex.com)
% License: CC BY-NC-SA 3.0 (http://creativecommons.org/licenses/by-nc-sa/3.0/)
% Modified by Alan G. Labouseur  - alan@labouseur.com
%
%%%%%%%%%%%%%%%%%%%%%%%%%%%%%%%%%%%%%%%%%

%----------------------------------------------------------------------------------------
%	PACKAGES AND OTHER DOCUMENT CONFIGURATIONS
%----------------------------------------------------------------------------------------

\documentclass[letterpaper, 10pt,DIV=13]{scrartcl} 

\usepackage[T1]{fontenc} % Use 8-bit encoding that has 256 glyphs
\usepackage[english]{babel} % English language/hyphenation
\usepackage{amsmath,amsfonts,amsthm,xfrac} % Math packages
\usepackage{sectsty} % Allows customizing section commands
\usepackage{graphicx}
\usepackage[lined,linesnumbered,commentsnumbered]{algorithm2e}
\usepackage{listings}
\usepackage{parskip}
\usepackage{lastpage}

\allsectionsfont{\normalfont\scshape} % Make all section titles in default font and small caps.

\usepackage{fancyhdr} % Custom headers and footers
\pagestyle{fancyplain} % Makes all pages in the document conform to the custom headers and footers

\fancyhead{} % No page header - if you want one, create it in the same way as the footers below
\fancyfoot[L]{} % Empty left footer
\fancyfoot[C]{} % Empty center footer
\fancyfoot[R]{page \thepage\ of \pageref{LastPage}} % Page numbering for right footer

\renewcommand{\headrulewidth}{0pt} % Remove header underlines
\renewcommand{\footrulewidth}{0pt} % Remove footer underlines
\setlength{\headheight}{13.6pt} % Customize the height of the header

\numberwithin{equation}{section} % Number equations within sections (i.e. 1.1, 1.2, 2.1, 2.2 instead of 1, 2, 3, 4)
\numberwithin{figure}{section} % Number figures within sections (i.e. 1.1, 1.2, 2.1, 2.2 instead of 1, 2, 3, 4)
\numberwithin{table}{section} % Number tables within sections (i.e. 1.1, 1.2, 2.1, 2.2 instead of 1, 2, 3, 4)

\setlength\parindent{0pt} % Removes all indentation from paragraphs.

\binoppenalty=3000
\relpenalty=3000

%----------------------------------------------------------------------------------------
%	TITLE SECTION
%----------------------------------------------------------------------------------------

\newcommand{\horrule}[1]{\rule{\linewidth}{#1}} % Create horizontal rule command with 1 argument of height

\title{	
   \normalfont \normalsize 
   \textsc{CMPT 424N-111 - Fall 2019 - Dr. Labouseur} \\[10pt] % Header stuff.
   \horrule{0.5pt} \\[0.25cm] 	% Top horizontal rule
   \huge Lab One  \\     	    % Assignment title
   \horrule{0.5pt} \\[0.25cm] 	% Bottom horizontal rule
}

\author{Eric Stenton \\ \normalsize Eric.Stenton1@Marist.edu}

\date{\normalsize\today} 	% Today's date.

\begin{document}
\maketitle % Print the title

%----------------------------------------------------------------------------------------
%   start PROBLEM ONE
%----------------------------------------------------------------------------------------
\section{Problem One}

\textbf{\emph{Question:}}
What are the advantages and disadvantages of using the same system call interface for manipulating both files and devices?

\textbf{\emph{Answer:}}
Using the same system call interface for manipulating both files and devices has both the advantages of a single interface to deal with the similar file and device system calls, but also the disadvantages of needing to be generalized to the extent that certain functionality may be lost. According to the textbook, the difference between files and I/O devices in terms of their commands/system calls is minimal to the extent that many operating systems such as UNIX merge them into a combined file-device structure\cite{concepts}. Due to this similarity, the more specific details of dealing with files and devices can be abstracted to use the same set of system calls and be interacted with in a similar way reducing development time. Such a design makes the file-device structure simple, albeit monolithic. The monolithic nature of the structure offers the disadvantage of not handling complexity and specific differences between files and devices. Since new devices can be added similar to files, future ones may require different system calls than the single system call interface can handle leading to a drop in certain functions or performance. In these cases, a separate system call interface with the supported system calls/commands would perform better than a combined file and device one.

%----------------------------------------------------------------------------------------
%   end PROBLEM ONE
%----------------------------------------------------------------------------------------

\pagebreak

%----------------------------------------------------------------------------------------
%   start PROBLEM TWO
%----------------------------------------------------------------------------------------
\section{Problem Two}

\textbf{\emph{Question:}}
Would it be possible for the user to develop a new command interpreter using the system call interface provided by the operating system? How?

\textbf{\emph{Answer:}}
Since the system call interface acts as the mediator between commands specified in a command interpreter and the related system calls that carry out the desired action in kernel mode, a user could indeed develop a new command interpreter that interacts with the existing system call interface of the OS. According to the textbook, there are two common command interpreter designs. The first design describes a command interpreter that houses the code for each command and support in communicating with the system call interface in its own code. The second describes a slimmer command interpreter that references a location where commands are stored as files and are called from there\cite{concepts}. The latter design is common in Linux machines. The user could develop a command interpreter of either design to interact with the system call interface since the user has access to this functionality in user mode. The Window's PowerShell is an example of a command interpreter that was later added to Microsoft's operating system that interacts with the system call interface similar to the Linux shell. 


%----------------------------------------------------------------------------------------
%   end PROBLEM Two
%----------------------------------------------------------------------------------------

%----------------------------------------------------------------------------------------
%   REFERENCES
%----------------------------------------------------------------------------------------
% The following two commands are all you need in the initial runs of your .tex file to
% produce the bibliography for the citations in your paper.
\bibliographystyle{abbrv}
\bibliography{lab01} 
% You must have a proper ".bib" file and remember to run:
% latex bibtex latex latex
% to resolve all references.

\end{document}
